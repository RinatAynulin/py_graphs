\documentclass{article}
\usepackage[T2A]{fontenc}
\usepackage[utf8]{inputenc}
\usepackage[english, russian]{babel}

\usepackage{geometry}
\geometry{a4paper, top=1cm, bottom=2cm, left=1.5cm, right=1.5cm}

\usepackage{indentfirst}
\usepackage{amssymb,amsfonts,amsmath}
\usepackage{cite,enumerate}
\usepackage[pdftex,colorlinks,unicode,bookmarks]{hyperref}

\usepackage{floatrow}
\usepackage{caption}
\usepackage{subcaption}
\usepackage{multirow}
\usepackage[table]{xcolor}

\title{
	Сравнение результатов вычислений моего кода и известных мне статей (анализ тестов)
}
\author{Владимир Ивашкин}

\begin{document}

\maketitle

\section*{Введение}
Захотелось написать большой развернутый отчет о всех тестах, которые я провожу. Возможно, это поможет найти наши ошибки и в будущем быть уверенными в результатах.

\section{Chebotarev: Studying new classes of graph metrics}
Ссылка: \url{https://arxiv.org/abs/1305.7514}

Здесь нам интересен Fig. 1. На графе "цепочка" можно прогнать такие же метрики при тех же параметрах. В обозначениях ниже я имею в виду, что вершины графа названы слева направо цифрами от 1 до 4. Расстояния здесь нормированы на то, чтобы сумма $D_{12} + D_{23} + D_{34} = 3$.
Достаточно будет сравнивать расстояния $D_{12}, D_{23}, D_{13}, D_{14}$.
Будем считать, что расстояния не соответствуют друг другу, если хотя бы одна соответствующая пара расстояний различается на 0.04 в абсолютной величине.

Вначале результаты не сходились, но потом выяснилось следующее:
\begin{itemize}
  \item В моем коде из всех ядер, перед тем, как превращать их в расстояния, брался корень. Кажется, мы обсуждали, что это нужно для Communicability, но в итоге это было включено везде. В этом причина, почему тесты не проходили
  \item Для Communicability это все-таки нужно, в этом случае все результаты совпадают
\end{itemize}

\begin{table}[H]
\centering
\caption{Тест Studying new classes of graph metrics, Figure 1}
\label{}
\begin{tabular}{rr|cccc}
             &      & $D_{12}$ & $D_{23}$ & $D_{13}$ & $D_{14}$ \\
             \hline
SP           & true & 1,000 & 1,000 & 2,000 & 3,000 \\
             & test & 1,000 & 1,000 & 2,000 & 3,000 \\
             \cline{2-6}
R            & true & 1,000 & 1,000 & 2,000 & 3,000 \\
             & test & 1,000 & 1,000 & 2,000 & 3,000 \\
             \cline{2-6}
Walk         & true & 1,025 & 0,950 & 1,975 & 3,000 \\
             & test & 1,025 & 0,950 & 1,975 & 3,000 \\
             \cline{2-6}
logFor       & true & 0,959 & 1,081 & 2,040 & 3,000 \\
             & test & 0,959 & 1,081 & 2,041 & 3,000 \\
             \cline{2-6}
For          & true & 1,026 & 0,947 & 1,500 & 1,895 \\
             & test & 1,026 & 0,947 & 1,500 & 1,895 \\
             \cline{2-6}
SqResistance & true & 1,000 & 1,000 & 1,414 & 1,732 \\
             & test & 1,000 & 1,000 & 1,414 & 1,732 \\
             \cline{2-6}
Comm         & true & 0,964 & 1,072 & 1,492 & 1,564 \\
             & test & 0,964 & 1,072 & 1,492 & 1,564 \\
             \cline{2-6}
pWalk 4.5    & true & 1,025 & 0,950 & 1,541 & 1,466 \\
             & test & 1,025 & 0,950 & 1,541 & 1,466 \\
             \cline{2-6}
pWalk 1.0    & true & 0,988 & 1,025 & 1,379 & 1,416 \\
             & test & 0,988 & 1,025 & 1,379 & 1,416
\end{tabular}
\end{table}

Также я воспроизвел результаты из Table 1 в Chebotarev: The Walk Distances in Graphs (ссылка: \url{https://arxiv.org/abs/1103.2059}). Скорее всего, они основаны на тех же результатах, что уже были в таблице выше, но почему бы нет.

\begin{table}[H]
\centering
\caption{The Walk Distances in Graphs, Table 1}
\label{my-label}
\begin{tabular}{rr|ccc}
          &      & $D_{12} / D_{23}$ & $(D_{12}+D_{23}) / D_{13}$ & $D_{14} / D_{12}$ \\
          \hline
SP        & true & 1,000       & 1,000               & 1,500       \\
          & test & 1,000       & 1,000               & 1,500       \\
          \cline{2-5}
R         & true & 1,000       & 1,000               & 1,500       \\
          & test & 1,000       & 1,000               & 1,500       \\
          \cline{2-5}
Walk      & true & 1,080       & 1,000               & 1,520       \\
          & test & 1,080       & 1,000               & 1,519       \\
          \cline{2-5}
logFor    & true & 0,890       & 1,000               & 1,470       \\
          & test & 0,887       & 1,000               & 1,470       \\
          \cline{2-5}
For       & true & 1,080       & 1,320               & 1,260       \\
          & test & 1,083       & 1,316               & 1,263       \\
          \cline{2-5}
pWalk 4.5 & true & 1,080       & 1,280               & 0,950       \\
          & test & 1,079       & 1,281               & 0,951       \\
          \cline{2-5}
pWalk 1.0 & true & 0,960       & 1,460               & 1,030       \\
          & test & 0,964       & 1,459               & 1,027      
\end{tabular}
\end{table}

Видим, что с этими тестами все ок. В последнем разделе я привожу сводную таблицу, где показываю, что именно было покрыто повторением результатов каждой статьи.

\section{Kivimaki: Developments in the theory of randomized shortest paths with a article comparison of graph node distances}
Ссылка: \url{https://arxiv.org/abs/1212.1666}

Здесь мы можем использовать два источника: это Figure 2, а также Table 2 с оптимальными значениями из Table 1.

\subsection{Figure 2}
Здесь исследуется поведение метрик RSP, FE, pRes, logFor, SP-CT при изменении их параметров в заданном интервале для графа "треугольник с хвостом". Можно исследовать всю кривую, но проще всего взять только крайние точки: слева отношение $\Delta_{12}/\Delta_{23}$ равно 1.5, справа --- 1.0.

После того, как я убрал взятие корня для logFor, все результаты сошлись:

\begin{table}[H]
\centering
\caption{Kivimaki, Figure 2}
\label{my-label}
\begin{tabular}{rll|cc|c}
      &       &       & \multicolumn{3}{c}{$D_{12} / D_{23}$} \\
border & measure & param & test        & true & diff   \\
      \hline
left  & CT &          & 1,5         & 1,5  & 0      \\
      & logFor & 500.0 & 1,4975      & 1,5  & 0,0025 \\
      & RSP & 0.0001   & 1,4992      & 1,5  & 0,0008 \\
      & FE & 0.0001    & 1,4996      & 1,5  & 0,0004 \\
      \hline
right & SP &          & 1           & 1    & 0      \\
      & logFor & 0.01  & 1,0011      & 1    & 0,0011 \\
      & RSP & 20.0     & 1           & 1    & 0      \\
      & FE & 20.0      & 0,9834      & 1    & 0,0166
\end{tabular}
\end{table}

\subsection{Table 2 с оптимальными значениями из Table 1}
Здесь проверяется качество (метрика качества --- NMI) кластеризации методом kMeans графов из датасета Newsgroups. Кернелы: RSP, FE, logFor, SP-CT, SCT. Результаты получаются похожими для всех метрик, кроме SP-CT. Результат очень плох: в статье ожидается качество порядка 70-80 NMI*100, по факту что SP, что CT дают 0.2-3 NMI*100. SP-CT применяется с параметром 1, то есть чистый SP. Видим проблему с SP.

\begin{table}[H]
\centering
\caption{Kivimaki, Table 2}
\label{my-label}
\begin{tabular}{rr|rrrrrr}
         &      & n2cl1 & n2cl2 & n2cl3 & n3cl1 & n3cl2 & n3cl3 \\
         \hline
RSP      & test & 79,443       & 57,914       & 81,070       & 77,092       & 76,797       & 75,520       \\
         & true & 84,500       & 58,700       & 81,000       & 76,600       & 77,000       & 76,500       \\
         & diff & 5,057        & 0,786        & 0,070        & 0,492        & 0,203        & 0,980        \\
         \cline{2-8}
FE       & test & 79,443       & 57,917       & 81,070       & 76,619       & 77,980       & 75,131       \\
         & true & 80,700       & 58,700       & 81,100       & 76,200       & 78,300       & 77,000       \\
         & diff & 1,257        & 0,783        & 0,030        & 0,419        & 0,320        & 1,869        \\
         \cline{2-8}
logFor H & test & 81,846       & 60,952       & 76,988       & 78,376       & 75,010       & 75,121       \\
         & true & 83,100       & 58,800       & 75,000       & 75,400       & 75,500       & 74,400       \\
         & diff & 1,254        & 2,152        & 1,988        & 2,976        & 0,490        & 0,721        \\
         \cline{2-8}
SP-CT K  & test & 0,219        & 0,147        & 0,201        & 0,315        & 0,334        & 0,295        \\
         & true & 65,200       & 51,200       & 85,900       & 74,200       & 62,600       & 71,500       \\
         & diff & \cellcolor{red!25} 64,981 & \cellcolor{red!25} 51,053 & \cellcolor{red!25} 85,699 &
                      \cellcolor{red!25} 73,885 & \cellcolor{red!25} 62,266 & \cellcolor{red!25} 71,205 \\
         \cline{2-8}
SCT H    & test & 81,105       & 54,616       & 78,440       & 77,922       & 72,276       & 75,409       \\
         & true & 81,600       & 56,800       & 79,600       & 77,300       & 73,000       & 75,900       \\
         & diff & 0,495        & 2,184        & 1,160        & 0,622        & 0,724        & 0,491       
\end{tabular}
\end{table}

Помимо статей я искал другие реализации мер для того, чтобы расширить количество тестов. Я наткнулся на вот эту реализацию: \url{https://github.com/jmmcd/GPDistance}. Здесь я увидел суть более сложные реализации RSP и FE. Насколько я понял, они защищены от случаев вроде тех, когда граф не связный. Я реализовал тесты из этого репозитория и увидел, что RSP и FE из этого репозитория работают стабильнее, чем мои варианты, сделанные строго по формулам из статей. Я заменил свои версии версиями из репозитория и они проходят тесты из статей. В частности, таблица выше содержит результаты с обновленными мерами.

\section{Sommer: Comparison of Graph Node Distances on Clustering Tasks}
Ссылка: (не находил в открытых источниках)

Здесь нас интересует Table 3 с оптимальными значениями из Table 2.
Метрики: CCT, FE, logFor, RSP, SCT, SP
Датасеты: football, newsgroups, polblogs, zachary
Проблемы: CCT не работает для football, на polblogs не работает ничего, видимо из-за большого размера. Для SP не проходят никакие тесты.

\begin{table}[H]
\centering
\caption{Sommer, Table 3}
\label{my-label}
\begin{tabular}{rr|rrrrrrrr}
       &      & n2cl1 & n2cl2 & n2cl3 & n3cl1 & n3cl2 & n3cl3 & zachary & football \\
       \hline
SCCT   & test & 0,794        & 0,598        & 0,758        & 0,784        & 0,758        & 0,746        & 1,000   & \cellcolor{red!25} error    \\
       & true & 0,794        & 0,582        & 0,758        & 0,778        & 0,762        & 0,746        & 1,000   &          \\
       & diff & 0,000        & 0,016        & 0,000        & 0,006        & 0,004        & 0,000        & 0,000   &          \\
       \cline{2-10}
FE     & test & 0,797        & 0,645        & 0,811        & 0,781        & 0,763        & 0,764        & 1,000   & 0,862    \\
       & true & 0,805        & 0,591        & 0,811        & 0,781        & 0,797        & 0,771        & 1,000   & 0,906    \\
       & diff & 0,008        & 0,054        & 0,000        & 0,000        & 0,034        & 0,006        & 0,000   & 0,045    \\
       \cline{2-10}
logFor & test & 0,831        & 0,622        & 0,769        & 0,746        & 0,745        & 0,752        & 1,000   & 0,895    \\
       & true & 0,838        & 0,584        & 0,748        & 0,753        & 0,758        & 0,749        & 1,000   & 0,903    \\
       & diff & 0,007        & 0,038        & 0,021        & 0,007        & 0,014        & 0,003        & 0,000   & 0,008    \\
       \cline{2-10}
RSP    & test & 0,797        & 0,635        & 0,785        & 0,781        & 0,786        & 0,725        & 1,000   & 0,895    \\
       & true & 0,797        & 0,580        & 0,796        & 0,781        & 0,776        & 0,730        & 1,000   & 0,909    \\
       & diff & 0,000        & 0,055        & 0,011        & 0,000        & 0,010        & 0,005        & 0,000   & 0,014    \\
       \cline{2-10}
SCT    & test & 0,820        & 0,625        & 0,824        & 0,753        & 0,723        & 0,765        & 1,000   & 0,845    \\
       & true & 0,817        & 0,552        & 0,786        & 0,773        & 0,728        & 0,763        & 1,000   & 0,811    \\
       & diff & 0,002        & 0,073        & 0,039        & 0,020        & 0,005        & 0,002        & 0,000   & 0,033    \\
       \cline{2-10}
SP     & test & 0,003        & 0,003        & 0,009        & 0,003        & 0,021        & 0,006        & 0,677   & 0,861    \\
       & true & 0,654        & 0,516        & 0,859        & 0,743        & 0,625        & 0,720        & 1,000   & 0,858    \\
       & diff & \cellcolor{red!25} 0,651 & \cellcolor{red!25} 0,513 & \cellcolor{red!25} 0,850 &
                    \cellcolor{red!25} 0,740 & \cellcolor{red!25} 0,603 & \cellcolor{red!25} 0,714 & \cellcolor{yellow!25} 0,323   & 0,004   
\end{tabular}
\end{table}

Похоже, все-таки у нас есть проблема.

Как работает SP:
\begin{itemize}
  \item Вызываю shortest\_path из scipy (визуально выдает правильные результаты)
  \item Применяется нормализация, чтобы параметр адекватно смешивал SP и CT
  \item Применяю $D \rightarrow K$ преобразование
\end{itemize}

Больше ничего тут нет. В оправдание функции shortest\_path скажу, что мы проверяли ее тестами выше. Но проблемы с $D \rightarrow K$ тоже быть не может, ведь RSP и FE преобразуются этой же формулой.
Виновата нормализация? Без нормализации наблюдаем ту же проблему. Что еще интересно, с уменьшением размеров графа качество кластеризации растет. Может, здесь даже проблема, как у Commute Time, описанная в Getting lost in space?

Я попробовал найти другие реализации shortest path --- не помогло. Попробовал найти сразу shortest path kernel и нашел здесь: \url{https://github.com/gmum/pykernels}, но результат все такой же плохой.

\section*{Общий результат}
Здесь я постараюсь записать в единый список всё, что мы используем в коде и наличие тестов.

\begin{table}[H]
\centering
\caption{Overall result}
\label{my-label}
\begin{tabular}{rr|cccc|c}
                & Subject                        & Chebotarev & Kivimaki & Sommer & Avrachenkov & Result \\
                \hline
Measure         & Shortest path                  & +          & \cellcolor{red!25} +/- & \cellcolor{red!25} +/- & & \cellcolor{red!25} - \\
                & Resistance            & +          & +        &        &             & \cellcolor{red!25} ? \\
                & SP-CT              &            & \cellcolor{red!25} - & & & \cellcolor{red!25} - \\
                & plain Walk                     & +          &          &        &             & +      \\
                & Walk                           & +          &          &        & +           & +      \\
                & Forest                         & +          &          &        &             & +      \\
                & logForest                      & +          & +        & +      & +           & +      \\
                & Comm                & +          &          &        &             & +      \\
                & logComm             &            &          &        & +           & +      \\
                & Heat                           &            &          &        &             & ?      \\
                & logHeat                        &            &          &        & +           & +      \\
                & SCT                            &            & +        & +      &             & +      \\
                & SCCT                           &            &          & +      &             & +      \\
                & RSP                            &            & +        & +      &             & +      \\
                & FE                             &            & +        & +      &             & +      \\
                & Normalized Heat                &            &          &        & +           & +      \\
                & P. PageRank          &            &          &        & +           & +      \\
                & Modified P. PageRank &            &          &        & +           & +      \\
                & Heat P. PageRank     &            &          &        & +           & +      \\
                \cline{2-7}
Transformation  & a -\textgreater t              & +          & +        &        &             & +      \\
                & H0 -\textgreater H             & +          & +        & +      &             & +      \\
                & H -\textgreater D              & +          & +        & +      &             & +      \\
                \cline{2-7}
Dataset         & News                           &            & +        & +      &             & +      \\
                & Football                       &            &          & +      &             & +      \\
                & Zachary                        &            &          & +      &             & +      \\
                \cline{2-7}
Graph Generator & Stochastic Block Model &            &          &        & +           & + \\
                \cline{2-7}
Clustering & KMeans & & + & + & + & +
\end{tabular}
\end{table}

Я не могу понять, что не так с простейшей метрикой. Может они используют не просто SP? В конце концов Kivimaki и Sommer могут иметь одну и ту же кодовую базу. 

\end{document}







