\documentclass{article}
\usepackage[T2A]{fontenc}
\usepackage[utf8]{inputenc}
\usepackage[english, russian]{babel}

\usepackage{geometry}
\geometry{a4paper, top=1cm, bottom=2cm, left=1.5cm, right=1.5cm}

\usepackage{indentfirst}
\usepackage{amssymb,amsfonts,amsmath}
\usepackage{cite,enumerate}
\usepackage[pdftex,colorlinks,unicode,bookmarks]{hyperref}

\usepackage{floatrow}
\usepackage{caption}
\usepackage{subcaption}
\usepackage{multirow}
\usepackage[table]{xcolor}

\title{
        Воспроизведение результатов известных статей в py\_graphs и анализ сравнения.
}
\author{Владимир Ивашкин}

\begin{document}

\maketitle


\section*{Введение}
Для того, чтобы проводить эксперименты с метриками, нам нужно быть уверенными, что метрики не содержат ошибок. В прошлом мы убеждались, что ошибки имеют место быть.
Я воспроизвел результаты четырех статей и оформил это в виде тестов к моему коду. Теперь после любых значимых изменений я буду запускать эти тесты и расследовать возникающие расхождения. Так победим.

Этот текст нужен в том числе и мне, чтобы не забыть, что именно я делал и чем руководствовался.


\section{Chebotarev: Studying new classes of graph metrics}
Ссылка: \url{https://arxiv.org/abs/1305.7514}

Здесь нам интересна Fig. 1. Здесь на графе "цепочка" показаны расстояния между вершинами в зависимости от метрики. Расстояния здесь нормированы на $D_{12} + D_{23} + D_{34} = 3$.
Достаточно будет сравнивать расстояния $D_{12}, D_{23}, D_{13}, D_{14}$.

Вначале результаты не сходились, но потом выяснилось следующее:
\begin{itemize}
  \item В моем коде из всех ядер, перед тем, как превращать их в расстояния, брался корень. Мы обсуждали, что это нужно для Communicability, но в итоге это было включено везде. В этом причина, почему раньше результаты не совпадали с этой работой;
  \item Для Communicability взятие корня все-таки нужно, в этом случае результаты совпадают по всем метрикам.
\end{itemize}

\begin{table}[H]
\centering
\caption{Тест Studying new classes of graph metrics, Figure 1}
\label{}
\begin{tabular}{rr|cccc}
             &      & $D_{12}$ & $D_{23}$ & $D_{13}$ & $D_{14}$ \\
             \hline
SP           & true & 1,000 & 1,000 & 2,000 & 3,000 \\
             & test & 1,000 & 1,000 & 2,000 & 3,000 \\
             \cline{2-6}
R            & true & 1,000 & 1,000 & 2,000 & 3,000 \\
             & test & 1,000 & 1,000 & 2,000 & 3,000 \\
             \cline{2-6}
Walk         & true & 1,025 & 0,950 & 1,975 & 3,000 \\
             & test & 1,025 & 0,950 & 1,975 & 3,000 \\
             \cline{2-6}
logFor       & true & 0,959 & 1,081 & 2,040 & 3,000 \\
             & test & 0,959 & 1,081 & 2,041 & 3,000 \\
             \cline{2-6}
For          & true & 1,026 & 0,947 & 1,500 & 1,895 \\
             & test & 1,026 & 0,947 & 1,500 & 1,895 \\
             \cline{2-6}
SqResistance & true & 1,000 & 1,000 & 1,414 & 1,732 \\
             & test & 1,000 & 1,000 & 1,414 & 1,732 \\
             \cline{2-6}
Comm         & true & 0,964 & 1,072 & 1,492 & 1,564 \\
             & test & 0,964 & 1,072 & 1,492 & 1,564 \\
             \cline{2-6}
pWalk 4.5    & true & 1,025 & 0,950 & 1,541 & 1,466 \\
             & test & 1,025 & 0,950 & 1,541 & 1,466 \\
             \cline{2-6}
pWalk 1.0    & true & 0,988 & 1,025 & 1,379 & 1,416 \\
             & test & 0,988 & 1,025 & 1,379 & 1,416
\end{tabular}
\end{table}

Также я воспроизвел результаты из Table 1 в Chebotarev: The Walk Distances in Graphs (ссылка: \url{https://arxiv.org/abs/1103.2059}). Скорее всего, они основаны на тех же результатах, что уже были в таблице выше, но почему бы нет.

\begin{table}[H]
\centering
\caption{The Walk Distances in Graphs, Table 1}
\label{my-label}
\begin{tabular}{rr|ccc}
          &      & $D_{12} / D_{23}$ & $(D_{12}+D_{23}) / D_{13}$ & $D_{14} / D_{12}$ \\
          \hline
SP        & true & 1,000 & 1,000 & 1,500 \\
          & test & 1,000 & 1,000 & 1,500 \\
          \cline{2-5}
R         & true & 1,000 & 1,000 & 1,500 \\
          & test & 1,000 & 1,000 & 1,500 \\
          \cline{2-5}
Walk      & true & 1,080 & 1,000 & 1,520 \\
          & test & 1,080 & 1,000 & 1,519 \\
          \cline{2-5}
logFor    & true & 0,890 & 1,000 & 1,470 \\
          & test & 0,887 & 1,000 & 1,470 \\
          \cline{2-5}
For       & true & 1,080 & 1,320 & 1,260 \\
          & test & 1,083 & 1,316 & 1,263 \\
          \cline{2-5}
pWalk 4.5 & true & 1,080 & 1,280 & 0,950 \\
          & test & 1,079 & 1,281 & 0,951 \\
          \cline{2-5}
pWalk 1.0 & true & 0,960 & 1,460 & 1,030 \\
          & test & 0,964 & 1,459 & 1,027 
\end{tabular}
\end{table}

Видим, что с этими тестами тоже все ок. В последнем разделе я привожу сводную таблицу, где показываю, что именно было покрыто воспроизведением результатов каждой статьи.


\section{Kivimaki: Developments in the theory of randomized shortest paths with a article comparison of graph node distances}
Ссылка: \url{https://arxiv.org/abs/1212.1666}

\subsection{Figure 2}
Здесь исследуется поведение метрик RSP, FE, pRes, logFor, SP-CT при изменении их параметров в заданном интервале для графа "треугольник с хвостом". Можно исследовать всю кривую, для простоты возьмем только крайние точки: слева отношение $\Delta_{12}/\Delta_{23}$ равно 1.5, справа --- 1.0.

Раньше здесь были проблемы у logFor но после того, как я перестал брать корень из матрицы расстояний, все результаты сошлись:

\begin{table}[H]
\centering
\caption{Kivimaki, Figure 2}
\label{my-label}
\begin{tabular}{rll|cc|c}
       &         &        & \multicolumn{3}{c}{$D_{12} / D_{23}$} \\
border & measure & param  & test   & true & diff   \\
       \hline
left   & CT      &        & 1,5    & 1,5  & 0      \\
       & logFor  & 500.0  & 1,4975 & 1,5  & 0,0025 \\
       & RSP     & 0.0001 & 1,4992 & 1,5  & 0,0008 \\
       & FE      & 0.0001 & 1,4996 & 1,5  & 0,0004 \\
       \hline
right  & SP      &        & 1      & 1    & 0      \\
       & logFor  & 0.01   & 1,0011 & 1    & 0,0011 \\
       & RSP     & 20.0   & 1      & 1    & 0      \\
       & FE      & 20.0   & 0,9834 & 1    & 0,0166
\end{tabular}
\end{table}

\subsection{Table 2 с оптимальными значениями из Table 1}
Здесь проверяется качество (по NMI*100) кластеризации методом kMeans графов из датасета Newsgroups. Кернелы: RSP, FE, logFor, SP-CT, SCT. Результаты совпадают со статьей для всех метрик, кроме SP-CT. Для последней результат очень плох: в статье ожидается качество порядка 70-80 NMI*100, по факту что SP, что CT дают 0.2-3 NMI*100. SP-CT применяется с параметром 1, то есть чистый SP.

\begin{table}[H]
\centering
\caption{Kivimaki, Table 2}
\label{my-label}
\begin{tabular}{rr|rrrrrr}
         &      & n2cl1  & n2cl2  & n2cl3  & n3cl1  & n3cl2  & n3cl3 \\
         \hline
RSP      & test & 79,443 & 57,914 & 81,070 & 77,092 & 76,797 & 75,520 \\
         & true & 84,500 & 58,700 & 81,000 & 76,600 & 77,000 & 76,500 \\
         & diff & 5,057  & 0,786  & 0,070  & 0,492  & 0,203  & 0,980  \\
         \cline{2-8}
FE       & test & 79,443 & 57,917 & 81,070 & 76,619 & 77,980 & 75,131 \\
         & true & 80,700 & 58,700 & 81,100 & 76,200 & 78,300 & 77,000 \\
         & diff & 1,257  & 0,783  & 0,030  & 0,419  & 0,320  & 1,869  \\
         \cline{2-8}
logFor H & test & 81,846 & 60,952 & 76,988 & 78,376 & 75,010 & 75,121 \\
         & true & 83,100 & 58,800 & 75,000 & 75,400 & 75,500 & 74,400 \\
         & diff & 1,254  & 2,152  & 1,988  & 2,976  & 0,490  & 0,721  \\
         \cline{2-8}
SP-CT K  & test & 0,219  & 0,147  & 0,201  & 0,315  & 0,334  & 0,295  \\
         & true & 65,200 & 51,200 & 85,900 & 74,200 & 62,600 & 71,500 \\
         & diff & \cellcolor{red!25} 64,981 & \cellcolor{red!25} 51,053 & \cellcolor{red!25} 85,699 &
                  \cellcolor{red!25} 73,885 & \cellcolor{red!25} 62,266 & \cellcolor{red!25} 71,205 \\
         \cline{2-8}
SCT H    & test & 81,105 & 54,616 & 78,440 & 77,922 & 72,276 & 75,409 \\
         & true & 81,600 & 56,800 & 79,600 & 77,300 & 73,000 & 75,900 \\
         & diff & 0,495  & 2,184  & 1,160  & 0,622  & 0,724  & 0,491  
\end{tabular}
\end{table}


\section{Sommer: Comparison of Graph Node Distances on Clustering Tasks}
Ссылка: (не находил в открытых источниках)

Здесь нас интересует Table 3 с оптимальными значениями из Table 2.
Метрики: CCT, FE, logFor, RSP, SCT, SP.
Датасеты: football, newsgroups, polblogs, zachary.
Проблемы: CCT не работает для football, на polblogs не работает ничего, видимо из-за большого размера. Для SP не проходят почти все тесты.

\begin{table}[H]
\centering
\caption{Sommer, Table 3}
\label{my-label}
\begin{tabular}{rr|rrrrrrrr}
       &      & n2cl1 & n2cl2 & n2cl3 & n3cl1 & n3cl2 & n3cl3 & zachary & football \\
       \hline
SCCT   & test & 0,794 & 0,598 & 0,758 & 0,784 & 0,758 & 0,746 & 1,000   & \cellcolor{red!25} error    \\
       & true & 0,794 & 0,582 & 0,758 & 0,778 & 0,762 & 0,746 & 1,000   &          \\
       & diff & 0,000 & 0,016 & 0,000 & 0,006 & 0,004 & 0,000 & 0,000   &          \\
       \cline{2-10}
FE     & test & 0,797 & 0,645 & 0,811 & 0,781 & 0,763 & 0,764 & 1,000   & 0,862    \\
       & true & 0,805 & 0,591 & 0,811 & 0,781 & 0,797 & 0,771 & 1,000   & 0,906    \\
       & diff & 0,008 & 0,054 & 0,000 & 0,000 & 0,034 & 0,006 & 0,000   & 0,045    \\
       \cline{2-10}
logFor & test & 0,831 & 0,622 & 0,769 & 0,746 & 0,745 & 0,752 & 1,000   & 0,895    \\
       & true & 0,838 & 0,584 & 0,748 & 0,753 & 0,758 & 0,749 & 1,000   & 0,903    \\
       & diff & 0,007 & 0,038 & 0,021 & 0,007 & 0,014 & 0,003 & 0,000   & 0,008    \\
       \cline{2-10}
RSP    & test & 0,797 & 0,635 & 0,785 & 0,781 & 0,786 & 0,725 & 1,000   & 0,895    \\
       & true & 0,797 & 0,580 & 0,796 & 0,781 & 0,776 & 0,730 & 1,000   & 0,909    \\
       & diff & 0,000 & 0,055 & 0,011 & 0,000 & 0,010 & 0,005 & 0,000   & 0,014    \\
       \cline{2-10}
SCT    & test & 0,820 & 0,625 & 0,824 & 0,753 & 0,723 & 0,765 & 1,000   & 0,845    \\
       & true & 0,817 & 0,552 & 0,786 & 0,773 & 0,728 & 0,763 & 1,000   & 0,811    \\
       & diff & 0,002 & 0,073 & 0,039 & 0,020 & 0,005 & 0,002 & 0,000   & 0,033    \\
       \cline{2-10}
SP     & test & 0,003 & 0,003 & 0,009 & 0,003 & 0,021 & 0,006 & 0,677   & 0,861    \\
       & true & 0,654 & 0,516 & 0,859 & 0,743 & 0,625 & 0,720 & 1,000   & 0,858    \\
       & diff & \cellcolor{red!25} 0,651 & \cellcolor{red!25} 0,513 & \cellcolor{red!25} 0,850 &
                    \cellcolor{red!25} 0,740 & \cellcolor{red!25} 0,603 & \cellcolor{red!25} 0,714 & \cellcolor{yellow!25} 0,323   & 0,004   
\end{tabular}
\end{table}

Все проблемы минорные, кроме SP. SP выдает плохое качество в обоих статьях.
Как работает SP:
\begin{itemize}
  \item Вызывается функция shortest\_path() из scipy (проверял на маленьких графах, выдает правильные результаты. Также были тесты по статье "Studying new classes of graph metrics", там тоже результаты верные)
  \item (опционально) Применяется нормализация, чтобы параметр адекватно смешивал SP и CT
  \item Применяется $D \rightarrow K$ преобразование
\end{itemize}

Больше ничего тут нет. Проблемы с $D \rightarrow K$ тоже быть не может, ведь RSP и FE преобразуются этой же формулой. Без нормализации наблюдаем ту же проблему. Если заменить kMeans на Ward, то качество тоже не растет --- значит проблема не специфична для кластеризатора.

 Что еще интересно, с уменьшением размеров графа качество кластеризации растет (видим, что на football получилось приличное качество). Может, здесь даже проблема, как у Commute Time, описанная в Getting lost in space? Но почему она не описана в статьях, по которым я делал тесты? Быть может, у них какой-то более хитрый SP?

Я попробовал найти другие реализации shortest path --- не помогло. Попробовал найти сразу shortest path kernel и нашел здесь: \url{https://github.com/gmum/pykernels}, но результат все такой же плохой.

Также искал другие реализации мер для того, чтобы расширить количество тестов. Наткнулся на вот этот репозиторий: \url{https://github.com/jmmcd/GPDistance}. Здесь есть более сложные реализации RSP и FE. Насколько я понял, они защищены от случаев вроде тех, когда граф не связный. Я реализовал тесты из этого репозитория и увидел, что RSP и FE из этого репозитория работают стабильнее, чем мои варианты, сделанные строго по формулам из статей. Я заменил свои версии версиями из репозитория и они проходят все наши тесты. В частности, таблицы выше содержит результаты с обновленными мерами.


\section{Avrachenkov: Kernels on Graphs as Proximity Measures}
Ссылка: \url{https://hal.inria.fr/hal-01647915/document}

Помимо статьи, здесь у наc был доступен код. Я добавил все метрики из этого кода к себе в репозиторий. Часть метрик у нас уже была реализована, часть --- нет.
Исследование можно разделить на две части: сравнение реализаций Рубанова и моих для совпадающих мер, и воспроизведение результатов кластеризации из статьи.

\subsection{Сравнение реализаций}
Сравнивались результаты для одного простого графа на всем пространстве параметров. Метрики: Walk, logComm, lohHeat, Forest. Метрики совпали с точностью  $0.0001$.

\subsection{Сравнение результатов}
Сравнивались результаты из секции "Balanced Model" для метрик Walk, logComm, lohHeat, Forest (мои реализации), а также Normalized Heat, Personalized PageRank, Modified Personalized PageRank, Heat Personalized PageRank (реализации Рубанова). Сравнение сделано для сгенерированных графов.

Вначале у результатов были расхождения: одни и те же реализации давали качество в среднем на 0.004 хуже, чем в статье. Дело оказалось в генераторе графов: несмотря на то, что в основе лежит одна и та же идея, реализации дают разные результаты. Насколько я понял, самое важное отличие --- они проверяют связность графа и подбирают только связные. С этим может быть связан результат лучше, чем для моего генератора. Используя реализацию генератора от Рубанова, получаем следующие результаты:

\begin{table}[H]
\centering
\caption{Balanced Model}
\label{my-label}
\begin{tabular}{r|rrrr}
                             & best param & test   & true   & diff   \\
                             \hline
Katz                         & 0,0057     & 0.0000 & 0.0072 & \cellcolor{red!25} 0.0072 \\
Estrada                      & 0,3292     & 0.0099 & 0.0084 & \cellcolor{yellow!25} 0.0015 \\
Heat                         & 0,8267     & 0.0000 & 0.0064 & \cellcolor{red!25} 0.0064 \\
NormalizedHeat               & 7,7917     & 0.0090 & 0.0066 & \cellcolor{yellow!25} 0.0024 \\
RegularizedLaplacian         & 2,7097     & 0.0091 & 0.0072 & \cellcolor{yellow!25} 0.0019 \\
PersonalizedPageRank         & 0.9632     & 0.0093 & 0.0073 & \cellcolor{yellow!25} 0.0020 \\
ModifiedPersonalizedPageRank & 0.9214     & 0.0092 & 0.0072 & \cellcolor{yellow!25} 0.0020 \\
HeatPersonalizedPageRank     & 2,931      & 0.0094 & 0.0074 & \cellcolor{yellow!25} 0.0020
\end{tabular}
\end{table}

\section*{Общий результат}

\begin{table}[H]
\centering
\caption{Overall result}
\label{my-label}
\begin{tabular}{rr|cccc|c}
                & Subject                  & Chebotarev & Kivimaki & Sommer & Avrachenkov & Result \\
                \hline
Measure         & Shortest path            & +          & \cellcolor{red!25} +/- & \cellcolor{red!25} +/- & & \cellcolor{red!25} - \\
                & Resistance               & +          & +        &        &             &  \cellcolor{yellow!25} +* \\
                & plain Walk               & +          &          &        &             & +      \\
                & Walk                     & +          &          &        & +           & +      \\
                & Forest                   & +          &          &        &             & +      \\
                & logForest                & +          & +        & +      & +           & +      \\
                & Comm                     & +          &          &        &             & +      \\
                & logComm                  &            &          &        & +           & +      \\
                & Heat                     &            &          &        &             & \cellcolor{yellow!25} ? \\
                & logHeat                  &            &          &        & +           & +      \\
                & SCT                      &            & +        & +      &             & +      \\
                & SCCT                     &            &          & +      &             & +      \\
                & RSP                      &            & +        & +      &             & +      \\
                & FE                       &            & +        & +      &             & +      \\
                & Normalized Heat          &            &          &        & +**         & \cellcolor{yellow!25} + \\
                & P. PageRank              &            &          &        & +**         & \cellcolor{yellow!25} + \\
                & Modified P. PageRank     &            &          &        & +**         & \cellcolor{yellow!25} + \\
                & Heat P. PageRank         &            &          &        & +**         & \cellcolor{yellow!25} + \\
                \cline{2-7}
Transformation  & $\alpha \rightarrow t$   & +          & +        &        &             & +      \\
                & $t / \rho$               & +          &          &        & +           & +      \\
                & $t / 2*(1 - t)$          & +          & +        & +      & +           & +      \\
                & $(1 - \beta) / \beta$    &            & +        & +      &             & +      \\
                & $H0 \rightarrow H$       & +          & +        & +      & +           & +      \\
                & $H \rightarrow D$        & +          & +        & +      &             & +      \\
                \cline{2-7}
Dataset         & Football                 &            &          & +      &             & +      \\
                & Polbooks                 &            &          &        &             & \cellcolor{yellow!25} ? \\
                & Polblogs                 &            &          & ?***   &             & \cellcolor{yellow!25} ? \\
                & Zachary                  &            &          & +      &             & +      \\
                & Newsgroup                &            & +        & +      &             & +      \\
                \cline{2-7}
Graph Generator & Stochastic Block Model   &            &          &        & \cellcolor{yellow!25} +/- & \cellcolor{yellow!25} + \\
                & Rubanov's implementation &            &          &        & +           & +      \\
                \cline{2-7}
Clustering      & kMeans                   &            & +        & +      &             & +      \\
                & Ward                     &            &          &        &             & \cellcolor{yellow!25} ? \\
                & Spectral                 &            &          &        & +           & +
\end{tabular}
\end{table}

{
\small
* На самом деле, тут уверенности нет. Мы видели, что проблемы с SP появились только на больших графах, возможно и у CT могут быть расхождения.

** Мы сравнили реализации Рубанова с его же результатами. Это не очень корректно, ведь если Рубанов ошибся в реализации мер, то мы этого не узнаем.

*** Датасет очень большой и на нем падают наши вычисления
}

\medskip

Я не могу понять, что не так с простейшей метрикой. Может они используют не просто SP? В конце концов Kivimaki и Sommer могут иметь одну и ту же кодовую базу.

Из хороших новостей: все остальные метрики можно считать покрытыми тестами. Можно быть уверенными, что с ними все хорошо.

\end{document}







